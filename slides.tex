\documentclass{beamer}

% Must be loaded first
\usepackage{tikz}

\usepackage[utf8]{inputenc}
\usepackage{textpos}

% Font configuration
\usepackage{fontspec}

\input{font.tex}

% Tikz for beautiful drawings
\usetikzlibrary{mindmap,backgrounds}
\usetikzlibrary{arrows.meta,arrows}
\usetikzlibrary{shapes.geometric}

% Minted configuration for source code highlighting
\usepackage{minted}
\setminted{highlightcolor=black!5, linenos}
\setminted{style=lovelace}

\usepackage[listings, minted]{tcolorbox}
\tcbset{left=6mm}

% Use the include theme
\usetheme{codecentric}

% Metadata
\title{Free All The Things}
\author{Markus Hauck}

% The presentation content
\begin{document}

\begin{frame}[noframenumbering,plain]
  \titlepage{}
\end{frame}

\section{Introduction}\label{sec:introduction}

\begin{frame}
\frametitle{Free All The Things}
\begin{itemize}
\item well known: free monads
\item maybe known: free applicatives
\item free monoids
\item free <you name it>
\end{itemize}
\end{frame}

\begin{frame}
  \frametitle{The Road Ahead}
\end{frame}

\begin{frame}
  \frametitle{What Is Free}
  A free functor is left adjoint to a forgetful functor
  What's the problem?
\end{frame}

\begin{frame}
  \frametitle{What Is Free}
  \begin{itemize}
  \item a free X is the minimal thing that satisfies X's laws
  \item \textbf{nothing} else!

  \end{itemize}
\end{frame}

\begin{frame}
  \frametitle{Why Free}
  \begin{itemize}
  \item having a Free X is good for a number of reasons
  \item use Free X as if it was X
  \item but the program is reified into some (data-)structure
  \item this structure can often be analyzed and optimized
  \item the killer: interpreters of the program can vary
  \end{itemize}
\end{frame}

\begin{frame}
  \frametitle{Scales of Power}
  \begin{itemize}
  \item the structures we will look at, are able to capture computations that have different power abilities
  \item monad: depend on previous values and branching
  \item applicative: fixed structure with arbitrary applicative effects in between
  \item functor: well\ldots
  \item monoid: limited power, but very flexible and composable
  \item surprise
  \end{itemize}
\end{frame}

\begin{frame}
  \frametitle{Free Vs Tagless}
  \begin{itemize}
  \item we will mostly look at the data structure version of Free X
  \item the alternative is to use finally tagless representations
  \end{itemize}
\end{frame}

\section{Freeing The Monad}\label{sec:free-monad}
\begin{frame}[fragile]
  \frametitle{Freeing The Monad}
  \begin{itemize}
  \item what are the operations?
  \end{itemize}
    \begin{center}
\begin{minted}{scala}
trait Monad[F[_]] {
  def pure[A](x: A): F[A]
  def flatMap[A, B](fa: F[A])(f: A => F[B]): F[B]
}
\end{minted}
\end{center}
\end{frame}

\begin{frame}[fragile]
  \frametitle{Freeing The Monad}
\begin{itemize}
  \item what are the laws?
\end{itemize}
    \begin{center}
\begin{minted}{scala}
// Left identity
pure(a).flatMap(f) === f(a)

// Right identity
fa.flatMap(pure) === fa

// Associativity
fa.flatMap(f).flatMap(g) ===
  fa.flatMap(a -> f(a).flatMap(g))
\end{minted}
    \end{center}
\end{frame}

\begin{frame}[fragile]
  \frametitle{Freeing The Monad}
  \begin{itemize}
  \item todo: the minimal ``thing'' that has a \textit{Monad} instance
    \textbf{satisfies} the laws
  \item simple idea: capture as data
  \item btw: any minimal combination works!
  \end{itemize}
\end{frame}

\begin{frame}[fragile]
  \frametitle{Freeing The Monad}
    \begin{center}
\begin{minted}{scala}
trait Monad[F[_]] {
  def pure[A](x: A): F[A]
  def flatMap[A, B](fa: F[A])(f: A => F[B]): F[B]
}
\end{minted}
\vspace{1cm}
\begin{minted}{scala}
sealed abstract class Free[F[_], A]

final case class Pure[F[_], A](a: A) extends Free[F, A]

final case class FlatMap[F[_], A, B](
  fa: Free[F, A], f: A => Free[F, B]) extends Free[F, B]
  \end{minted}
\end{center}
\end{frame}

\begin{frame}[fragile]
  \frametitle{Freeing The Monad}
    \begin{center}
\begin{minted}{scala}
implicit def freeMonad[F[_], A]: Monad[Free[F, A]] =
  new Monad[Free[F, A]] {
    def pure[A](x: A): Free[F, A] = Pure(x)

    def flatMap[A, B](fa: Free[F, A])(
      f: A => Free[F, B]): F[B] = FlatMap(fa, f)
  }
\end{minted}
\end{center}
\end{frame}

\begin{frame}
  \frametitle{Freeing The Monad}
  \begin{itemize}
  \item but what about the laws?!
  \item clearly we are violating all of them!
  \item we need one more thing: the interpreter
  \end{itemize}
  \begin{minted}{scala}
def runFree[F[_], M[_]:Monad, A](nat: FunctionK[F, M]): Free[F, A] => M[A] = ???
  \end{minted}
\end{frame}

\section{Freeing The Applicative}\label{sec:free-applicative}

\begin{frame}
  \frametitle{Freeing The Applicative}
  \begin{itemize}
  \item we follow the same pattern
  \end{itemize}
\end{frame}

\section{Second Part}

\begin{frame}
  \frametitle{Mind Maps}
  \begin{center}
    \resizebox{0.8\textwidth}{!}{\begin{tikzpicture}
  \path[small mindmap, concept color=beamer@codeblue,
  level 1 concept/.append style={every child/.style={concept color=beamer@centricgreen}},
  level 2 concept/.append style={every child/.style={concept color=beamer@centricgreen!50}}
  ]
  node[concept] {Computer Science}
  [clockwise from=0]
  child {
    node[concept] {practical}
    [clockwise from=90]
    child { node[concept] (algo) {algorithms} }
    child { node[concept] {data structures} }
    child { node[concept] {pro\-gramming languages} }
    child { node[concept] {software engineer\-ing} }
  }
  child { node[concept] {technical} }
  child { node[concept] (theo) {theoretical} }
  child {
    node[concept] {applied}
    [clockwise from=180]
    child { node[concept] {databases} }
    child { node[concept] {WWW} }
  }
  
  node [annotation, xshift=-2cm, yshift=-5mm] (annot) at (theo.south) {
    \setsansfont{Caveat} \large This is an annotation }
  node [annotation, xshift=1cm, yshift=15mm] (annot2) at (algo.north) {
    \setsansfont{Caveat} \large This is another annotation }
  ;

  \begin{pgfonlayer}{background}
    \draw[draw=black, thick, shorten <=1mm, shorten >=1mm, -{Stealth[length=3mm, open, round]}] (annot) edge (theo);
    \draw[draw=black, thick, shorten <=1mm, shorten >=1mm, -{Stealth[length=3mm, open, round]}] (annot2) edge (algo);
  \end{pgfonlayer}
\end{tikzpicture}}
  \end{center}
\end{frame}

\begin{frame}
  \frametitle{Colorful Boxes}
  \begin{tcolorbox}[
    fonttitle=\sffamily\bfseries\large,
    colbacktitle=black,
    colframe=black,
    coltitle=beamer@centricgreen,
    title=Hello World
    ]
    This is a \textbf{tcolorbox}.
  \end{tcolorbox}
  \begin{tcolorbox}[
    fonttitle=\sffamily\bfseries\large,
    colbacktitle=black,
    colframe=black,
    coltitle=beamer@codeblue,
    title=Hello World
    ]
    This is a \textbf{tcolorbox}.
  \end{tcolorbox}
  \begin{tcolorbox}[
    fonttitle=\sffamily\bfseries\large,
    colbacktitle=black,
    colframe=black,
    title=Hello World
    ]
    This is a \textbf{tcolorbox}.
  \end{tcolorbox}
\end{frame}

\section{Conclusion}\label{sec:conclusion}

\begin{frame}
  \begin{center}
    \huge
    Your conclusion here
  \end{center}
\end{frame}

\end{document}
